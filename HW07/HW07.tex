\documentclass[12pt,letterpaper]{hmcpset}
\usepackage[margin=1in]{geometry}
\usepackage{graphicx}
\usepackage{amssymb}
\newcommand\tab[1][1cm]{\hspace*{#1}}
% info for header block in upper right hand corner
\name{Name: \underline{\hspace{4cm}}}
\class{Math 40, Section \underline{\hspace{1cm}}}
\assignment{HW07}
\duedate{2/28/18}

\begin{document}
\section*{}
\problemlist{Homework 07: EigenThings and Diagonalization \\4.3 \#10, 22, 25, 33, 37 \\ 4.4 \#7, 19, 42, 49, 51}

%------------------------- Problem 0 -----------------------

\begin{problem}[4.3.10]
    Compute (a) the characteristic polynomial of A, (b) the eigenvalues of A, (c) the basis for each eigenspace of A, and (d) the algebraic and geometric multiplicity of each eigenvalue.
    \[A=\begin{bmatrix}
    2 & 1 & 1 & 0 & \\
    0 & 1 & 4 & 5 & \\
    0 & 0 & 3 & 1 & \\
    0 & 0 & 0 & 2 & \\
    \end{bmatrix}
    \]
\end{problem}

\begin{solution}
    \vfill
\end{solution}

\newpage
%------------------------- Problem 1 -----------------------

\begin{problem}[4.3.22]
    If $\vec{v}$ is an eigenvector of $A$ with corresponding eigenvalue $\lambda$ and $c$ is a scalar, show that $\vec{v}$ is an eigenvector of $A - cI$ with a corresponding eigenvalue $\lambda - c$.
\end{problem}

\begin{solution}
    \vfill
\end{solution}

\newpage
%------------------------- Problem 2 -----------------------

\begin{problem}[4.3.25]
    If $A$ and $B$ are two row equivalent matrices, do they necessarily have the same eigenvalues? Either prove that they do or give a counterexample.
\end{problem}

\begin{solution}
    \vfill
\end{solution}

\newpage
%------------------------- Problem 3 -----------------------

\begin{problem}[4.3.33]
    Verify the Cayley-Hamilton Theorem for $A=\begin{bmatrix}
    1 & -1 \\ 
    2 & 3
    \end{bmatrix} $. That is, find the characteristic polynomial $c_A(\lambda)$ of $A$ and show that $c_A(A) = O$
\end{problem}

\begin{solution}
    \vfill
\end{solution}

\newpage
%------------------------- Problem 4 -----------------------

\begin{problem}[4.3.37]
    For the matrix $A$ in Exercise 33, use the Cayley-Hamilton Theorem to compute $A^{-1}$ by expressing it as a linear combination of $I$, $A$, and $A^2$.
\end{problem}

\begin{solution}
    \vfill
\end{solution}

\newpage
%------------------------- Problem 5 -----------------------

\begin{problem}[4.4.7]
    A diagonalization of the matrix $A$ is given in the form $P^{-1}AP=D$. List the eigenvalues of $A$ and bases for the corresponding matrices.
    \[
    \begin{bmatrix}
    \frac{1}{8} & \frac{1}{8} & \frac{1}{8} \\ 
    -\frac{1}{4} & \frac{3}{4} & -\frac{1}{4} \\ 
    \frac{5}{8} & -\frac{5}{8} & -\frac{3}{8}
    \end{bmatrix} 
    \]
\end{problem}

\begin{solution}
    \vfill
\end{solution}

\newpage
%------------------------- Problem 6 -----------------------

\begin{problem}[4.4.19]
    Give $A^k$ as a product of three matrices.
    \[
    \begin{bmatrix}
    0 & 3 \\ 
    1 & 2
    \end{bmatrix}^k
    \]
\end{problem}

\begin{solution}
    \vfill
\end{solution}

\newpage
%------------------------- Problem 7 -----------------------

\begin{problem}[4.4.42]
    Prove that if $A$ is similar to $B$, then $A^T$ is similar to $B^T$.
\end{problem}

\begin{solution}
    \vfill
\end{solution}

\newpage
%------------------------- Problem 8 -----------------------

\begin{problem}[4.4.49]
    Prove that if $A$ is a diagonalizable matrix such that every eigenvalue of $A$ is either 0 or 1, then $A$ is idempotent (that is, $A^2=A$).
\end{problem}

\begin{solution}
    \vfill
\end{solution}

\newpage
%------------------------- Problem 9 -----------------------

\begin{problem}[4.4.51]
    Suppose that $A$ is a 6 $\times$ 6 matrix with characteristic polynomial
    \[c_A(\lambda)=(l+\lambda)(1-\lambda)^2(2-\lambda)^3\].
    \\
    (a) Prove that it is not possible to find linearly independent vectors $\vec{v_1}$, $\vec{v_2}$, and $\vec{v_3}$ in $\mathbb{R}^6$ such that $A\vec{v_1}=\vec{v_1}$, $A\vec{v_2}=\vec{v_2}$, and $A\vec{v_3}=\vec{v_3}$.
    \\\\
    (b) If $A$ is diagonalizable, what are the dimensions of the eigenspaces $E_{-1}$, $E_1$, and $E_2$?
\end{problem}

\begin{solution}
    \vfill
\end{solution}

\newpage
\end{document}