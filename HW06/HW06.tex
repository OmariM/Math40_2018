
\documentclass[12pt,letterpaper]{hmcpset}
\usepackage[margin=1in]{geometry}
\usepackage{graphicx}
\usepackage{amssymb}
\usepackage{gensymb}
% info for header block in upper right hand corner
\name{Name: \underline{\hspace{3cm}}}
\class{Math 40, Section \underline{\hspace{1cm}}}
\assignment{HW 06; Determinants and Intro to EigenThings }
\duedate{February 23, 2018}

\begin{document}
\section*{}



\problemlist{Section 4.2 Numbers; 14,24,52,54,56 \\ Exploration Number 10 \\ Section 4.1 Numbers; 6, 12, 14, 20, 26\\ Section 4.3 Numbers; 6a, 6b, 18  }

%------------------------- Problem 1 -----------------------

\begin{problem}[4.2.14]
\textit{Compute the determinant using cofactor expansion along any row or column that seems convenient}
$$
\begin{vmatrix}
2&0&3&-1\\
1&0&2&2\\
0&-1&1&4\\
2&0&1&-3
\end{vmatrix}
$$
\end{problem}

\begin{solution}
\end{solution}

\newpage

%------------------------- Problem 2 -----------------------


\begin{problem}[4.2.24]
\textit{Evaluate the given determinant using elementary row and/or column operations and Theorem 4.3 to reduce the matrix to row echelon form. }
$$
\begin{vmatrix}
1&-1&0&3\\
2&5&2&6\\
0&1&0&0\\
1&4&2&1
\end{vmatrix}
$$
\end{problem}
\begin{solution}
\end{solution}

\newpage

%------------------------- Problem 3 -----------------------


\begin{problem}[4.2.52]
\textit{Assume that A and B are $ n\times n$ matrices with $det A =3 $ and $det B = -2$.}
\begin{center}
\textbf{\textit{What is $det(AA^T)$?}}
\end{center}
\end{problem}

\begin{solution}
\end{solution}

\newpage

%------------------------- Problem 4 -----------------------


\begin{problem}[4.2.54]
\textit{A and B are $n\times n$ matrices;}
\begin{center}
\textit{If B is invertible, prove that $det (B^{-1}AB)= det(A)$}
\end{center}
\end{problem}

\begin{solution}
\end{solution}

\newpage
%------------------------- Problem 5 -----------------------


\begin{problem}[4.2.56]
\textit{A and B are $n\times n$ matrices;}
\begin{center}
\textit{ A square matrix A is called \textbf{nilpotent} if $A^m = 0$ for some $m > 1$ (The word nilpotent comes from the Latin, nil, meaning "nothing" and potere, meaning "to have power". A nilpotent matrix is thus one that that becomes "nothing" - when raised to some power) Find all possible values of det(A) if A is nilpotent.}
\end{center}
\end{problem}

\begin{solution}
\end{solution}

\newpage

%------------------------- Problem 6 -----------------------

\begin{problem}[Exploration 10]

\textit{Let A be a $3\times3$ matric and let P be the parallelepiped determined by the vectors \textbf{u}, \textbf{v}, and \textbf{w}. Let $T_A(P)$ denote the parallelpiped determined by $T_A(\textbf{u})= A\textbf{u}$ , $T_A(\textbf{v})= A\textbf{v}$ , and $T_A(\textbf{w})= A\textbf{w}$. Prove that the volume of $T_A(P)$ is given by $| det A|$ (volume of P)}
\end{problem}
\begin{solution}
\end{solution}

\newpage

%------------------------- Problem 7 -----------------------


\begin{problem}[4.1.6]

\textit{Show that \textbf{v} is an eigenvector of A and find the corresponding eigenvalue }
$$
A=\begin{bmatrix}
0&1&-1\\
1&1&1\\
1&2&0\\
\end{bmatrix}
, \textbf{v}= \begin{bmatrix}
-2\\1\\1
\end{bmatrix}
$$
\end{problem}
\begin{solution}
\end{solution}

\newpage


%------------------------- Problem 8 -----------------------


\begin{problem}[4.1.12]
\textit{Show that $\lambda$ is an eigenvalue of A and find one eigenvector corresponding to this eigenvector corresponding to this eigenvalue;}
$$
A=\begin{bmatrix}
3&1&-1\\
1&1&1\\
4&2&0
\end{bmatrix}
, \lambda=2
$$
\end{problem}

\begin{solution}
\end{solution}

\newpage

%------------------------- Problem 9 -----------------------


\begin{problem}[4.1.14]
\textit{Find the eigenvalues and eigenvectors of A geometrically} 
$$ A=\begin{bmatrix}
0&1\\1&0
\end{bmatrix} (\textit{reflection in the line } y=x)
$$
\end{problem}

\begin{solution}
\end{solution}

\newpage

%------------------------- Problem 10 -----------------------


\begin{problem}[4.1.20]
\textit{The unit vectors \textbf{x} in $\mathbb{R}^2$ and their images A\textbf{x} under the action of a $2\times 2$ matrix A are drawn head-to-tail as in Figure 4.7/ Estimate the eigenvectors and eigenvalues of A from each "eigenpicture"}
\end{problem}




\begin{solution}

\end{solution}

\newpage

%------------------------- Problem 11 -----------------------


\begin{problem}[4.1.26]
\textit{Using the method of Example 4.5 to find all of eigenvalues of the matrix A. Give bases for each of the corresponding eigenspaces. Illustrate the eigenspaces and the effect of multiplying eigenvectors by A as in Figure 4.8}
$$ A = \begin{bmatrix}
1&2\\-2&3
\end{bmatrix}
$$
\end{problem}

\begin{solution}
\end{solution}

\newpage

%------------------------- Problem 12 -----------------------


\begin{problem}[4.3.6 ]
\textit{(a) Compute the characteristic polynomial of A and (b) the eigenvalues of A;}
$$
A = \begin{bmatrix}
1&0&2\\
3&-1&3\\
2&0&1\\
\end{bmatrix}
$$
\end{problem}

\begin{solution}
\end{solution}

\newpage

%------------------------- Problem 13 -----------------------

\begin{problem}[4.3.18]
\begin{center}
\textit{A is a $3\times3$ matrix with eigenvectors;}
	$$ \textbf{v}_1 = \begin{bmatrix}
	1\\0\\0
	\end{bmatrix} 
	\textbf{v}_2 = \begin{bmatrix}
	1\\1\\0
	\end{bmatrix} \textit{, and }
	\textbf{v}_3 = \begin{bmatrix}
	1\\1\\1
	\end{bmatrix}
	$$ 
	\textit{ with corresponding eigenvalues $\lambda_1= -\frac{1}{3}$ , $\lambda_2= \frac{1}{3}$ , and $\lambda_3= 1$, respectivley, and;}
	$$\textbf{x}= \begin{bmatrix}
	2\\1\\2
\end{bmatrix}
$$
\textit{Find $A^k\textbf{x}$ What happens as $k$ becomes large( ie., $k \rightarrow \infty	$)} 
\end{center}
\end{problem}

\begin{solution}
\end{solution}

\newpage

\end{document}
